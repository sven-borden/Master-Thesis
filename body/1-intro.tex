%%INTRODUCTION
\chapter{Introduction}
\setcounter{page}{1}

\section{Multistable mechanisms}
    Working with a robot based on multistable joint require to have some understanding of multistable and bistable mechanism in first place. A multistable mechanism is able to hold more than one stable state. This imply that these mechanisms can maintain different deformations with zero force \cite{programmable_multistable_mechanism}. This principle is very interesting for different application where the resting energy needs to be maintained low or near zero. Example of bistable mechanical devices exists in the literature and find application in valves, switches, closures and clasps \cite{howell}. Bistable mechanisms needs to be able to store and release energy, therefore they typically involves combination of springs, which imply having stable position occurring during local minima of stored energy. Bistable mechanism includes for example a double-slider with a link joining the sliders \cite{howell}.
    Multistable mechanisms halve also been explored using origamis style robot. Multi-stable origami robots are attractive from multiple points of view, as they can be small and scalable to deploy adaptive structures \cite{Pagano_2017}. The multi-stability of the origami robot is based on a mix of stiff modules and soft modules that are part of the same structure. The uses of the origami structure was created to mimic the crawling locomotion. Similar studies have also explored the multistable origami mechanism approach to generate locomotion gaits without the need of complex controllers \cite{peristaltic}.\\
    
    In our case, a specific type of bistable joint was developed, involving two blocks linked together by rigid arms and a spring as seen on Figure \ref{fig:joint_basics}. This mechanism can produces one or two stable position for the top block depending on the spring's anchor position. The interesting point of this mechanism is that it is stackable vertically to increase the number of stable points. If we add a third block to our bistable structure and link it with two rigid arms and a new spring. We can create a multistable mechanism that has up to four stable positions. It is interesting to look how this multistable structure behaves when we attach an actuator to the top block and we apply a horizontal displacement. It can be demonstrated that depending on the anchors positions, the sequence of block displacement will be different. For example a sequence where in the forward motion, the top block will move before the middle block can be created, as well as having the middle block moving before the top block. 
    
    Multiple sequences can therefore exist depending on the specific setup selected. Interestingly, if there is a connection between the two moving block of the multistable structure, there is a possibility to create a \textit{leg} that will displace with a specific pattern depending the sequence selected. This pattern generally introduce a hysteresis, meaning the path followed during a forward input is different than the path followed during a backward input. This asymmetry added for a four legged robot can produce interesting results of locomotion using only two main actuators to control the robot. As this robot can provide interesting knowledge by using only two actuators to control the robot position and orientation on a surface, it gives hard challenge to understand how the different sequences impact the robot's behavior. This is the reason why it is necessary to create a numeric model of the robot and its legs to generate different simulations that could predict the displacement of the robot. Once the robot's behavior characterized, it would be easier to create a specific controller that would not only target this robot but more generally multistable based robots as most of them will works by transiting from one stable state to another. Specifically it will be interesting to use a Reinforcement learning controller to drives the robot around a virtual environment.
    

\section{Objectives}
    The objective of this thesis is to create a workflow that build a controller for a multistable based robot. In involves the following points:
    \begin{itemize}
        \item Simulate the legs patterns for the possible multistable sequences.
        \item Simulate the robot's behavior when actuators are running (for different sequences).
        \item Create a controller for the multistable robot.
    \end{itemize}

    Those objectives will allows to have a complete set of tools to analyze and create a controller for at least our multistable robot, but which could be extended to other robots that are based on multistable joints.\\